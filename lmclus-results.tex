\subsection{Understanding the Results of Linear Manifold Clustering}
\label{ssec:lmclus-results}

%TODO: wrong \lambda_n^2 < \theta^2
Let the measurement space be $R^N$ and the data set, consisting of
a list of observations from $R^N$, be $D$. A linear manifold cluster
$C$ of dimension $K$ is defined by
\begin{eqnarray*}
C=\{x\in D\ &|&\ \mbox{ for some }\lambda_1,\ldots,\lambda_N,\\
 && x=\mu+\sum_{n=1}^N \lambda_n\alpha_n \mbox{ where }\\
&&      \sum_{k=K+1}^N \lambda_n^2 < \theta^2\}
\end{eqnarray*}

$\mu$ is the center of the cluster so it is the vector that translates
the origin to the center of the manifold. It is computed as the mean of the
data vectors that are in the cluster.

\begin{eqnarray*}
\mu&=&\frac{1}{|C|}\sum_{x\in C} x
\end{eqnarray*}


$\alpha_1,\ldots,\alpha_N$ are an orthonormal basis for $R^N$ and
$\alpha_1,\ldots,\alpha_K$ constitute the basis for the $K$-dimensional
manifold.

For any $x\in C$, the corresponding $\lambda_1,\ldots,\lambda_N$ are given
by
\begin{eqnarray*}
\lambda_n&=&\alpha_n^\prime(x-\mu)
\end{eqnarray*}

The linear manifold bounding box $B$ is given by

\begin{eqnarray*}
B=\{x\in R^N\ &|&\ \mbox{ for some } \lambda_1,\ldots,\lambda_K,
 x=\mu+\sum_{k=1}^K \lambda_k\alpha_k \\
 \mbox{ where }\lambda_{kmin}\le&\lambda_k& \le \lambda_{kmax} \\
\lambda_{kmin}&=&\min_{x\in C} \alpha_k^\prime(x-\mu), k=1,\ldots,K\\
\lambda_{kmax}&=&\max_{x\in C} \alpha_k^\prime(x-\mu), k=1,\ldots,K\}\\
\end{eqnarray*}

The constants $\lambda_{kmin}$ and $\lambda_{kmax}$ determine the extent
of the boundary of the bounding box in dimension $k$ whose basis vector is $\alpha_k$.

To show the distribution of the points in the cluster, we show the histogram
of the data orthogonally projected onto each basis vector of the cluster.
Let $D_k$ be list of the projections on the $k^{th}$ axis. Then
\begin{eqnarray*}
D_k&=&< a\in [\lambda_{kmin},\lambda_{kmax}] \ | \ \mbox{ for some } x\in C,\ a=\alpha_k^\prime(x-\mu)>
\end{eqnarray*}

If the histogram of the $k^{th}$ axis has $M$ intervals, $[\eta_m,\eta_{m+1}], m=1\ldots,M$
then
\begin{eqnarray*}
\eta_m&=&\lambda_{kmin}+ \frac{(m-1)}{M}(\lambda_{kmax}-\lambda_{kmin})
\end{eqnarray*}
The probability $p_m$ that a data point of the cluster is in interval $[\eta_m,\eta_{m+1}]$
is given by
\begin{eqnarray*}
p_m&=&\frac{1}{\#D_k} \# < x\in C\ | \ \eta_m \le \alpha_k^\prime(x-\mu) \le \eta_{m+1}>
\end{eqnarray*}

The $M$ heights of the histogram $H_k$ of the $k^{th}$ axis is then given by
$p_1,\ldots,p_M$.
These $K$ histograms then give a first order picture of how the data is distributed
on the manifold in the reference frame of the manifold.

Of course the observed data points are not, in general, on the cluster manifold.
Rather, they are close to the manifold. The distance $\rho$ of an observed data point
$x$ from the manifold is given by

\begin{eqnarray*}
\rho(x)&=&\sqrt{\sum_{k=K+1}^N [\alpha_k^\prime (x-\mu)]^2}\\
&=& \sqrt{(x-\mu)^\prime(x-\mu)-\sum_{k=1}^K[\alpha_k^\prime (x-\mu)]^2}
\end{eqnarray*}

If there are $M$ intervals to the histogram and the $m^{th}$ interval
is $[\delta_m,\delta_{m+1}]$, then

\begin{eqnarray*}
\delta_m=\rho_{min}+\frac{m-1}{M}(\rho_{max}-\rho_{min})
\end{eqnarray*}
where
\begin{eqnarray*}
\rho_{min}&=&\min_{x\in C}\rho(x)\\
\rho_{max}&=&\theta
\end{eqnarray*}

The heights $q_1,\ldots,q_M$ of the histogram $H$ are given by

\begin{eqnarray*}
q_m&=&\frac{1}{\#C}<x\in C\ | \ \delta_m \le \rho(x) \le \delta_{m+1},\ m=1,\ldots,M>
\end{eqnarray*}
This histogram shows the spread of the data off the manifold.

To complete the information of the cluster, the vectors
\begin{itemize}
\item $\mu$
\item $\alpha_1,\ldots,\alpha_K$
\end{itemize}
must be given.

For those people who would like to understand the linear manifold cluster
in terms of a set of constraint equations, define

\begin{eqnarray*}
B&=&\left(\begin{array}{c}
               \alpha_{K+1}^\prime\\
               \alpha_{K+2}^\prime\\
               \vdots\\
               \alpha_{N}^\prime
               \end{array}\right)
\end{eqnarray*}

Then the constraint equation for the manifold of the cluster is given by

\begin{eqnarray*}
B(x-\mu)&=&0
\end{eqnarray*}

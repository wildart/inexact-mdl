\section{Conclusion}
\label{sc:conclusion}

We described a novel regularization technique for the linear manifold clustering
based on the idea that a linear manifold shaped cluster allows efficient compression
of the cluster data to the degree allowed by the specified error threshold.
This intuitive criterion was formalized as the minimization problem of
the description length of a prospective cluster and incorporated into
the stochastic search of the clustering algorithm.

In the empirical part of the work we studied the behavior of the proposed MDL
encoding, and the effect of the quantization error on it. We confirmed that
the described method produced reasonable results for simulated datasets, as
well as on the climate data clustering task. We believe that this regularization
technique allows creation of clusters that are more informative and
comprehensive.

A comprehensive scoring of the clusters with MDL values provides not only a
criteria for cluster goodness-of-fit evaluation, but as well can be viewed as
a qualitative measure which is used to explore stability of the clustering
algorithm \cite{VonLuxburg:2005VB}, or to improve clustering performance by
introducing a scoring function for a guided stochastic search, which is left to
be explored in our future research.
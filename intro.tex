% -*- root: inexact_mdl_lmc.tex -*-
\section{Introduction}\label{sc:intro}

Minimum Description Length (MDL) can be understood to be a technical specification
or formalization of the Occam's razor principle to understand a data set.
One way to pose the general problem is given a data set, define a language in
which to represent the data set so that the data set described in the language
is a meaningful description and the number of bits for the representation in
the language is minimal. This is different from data compression in the sense
that data compression only uses information theoretic methods to minimize the
number of bits to represent the data set, but the representation in itself
is not meaningful. It does not give any insight into the structure of the data.
\IfClass{IEEEtran}{}{
By first defining a language which is capable of representing the data and
the structure of the data in a meaningful way, minimum description length does
compress the data and simultaneously provides the structure of the data as well.
}
If there are multiple possible languages for describing the data, minimum
description length can be the principle for deciding which is the best language.

\IfClass{IEEEtran}{}{
It is easy to give an example of data that has a structure. Consider a data set
upon which it is appropriate to do linear regression. So there are $K$ observations
of the values of $N$ independent variables in which the values are considered
to have no error. For each of the $K$ observations, there is one dependent
variable whose value is considered to be noisy. The linear regression context
assumes that the dependent variable is a linear combination of the $N$ independent
variables. This linear combination is the structure of the data. However,
regression only provides the estimates of the linear combination and the total
squared error. So we have the structure of the data but we do not know the values
of the $K$ observations of the dependent variable. An MDL approach would define
the language to be the language of linear combinations. It would assume that
the $K$ observations of the $N$ independent variables are known precisely and
do not enter into the MDL description. It would then represent the $K$ observations
of the dependent variable. But instead of representing these $K$ observations
independent of the linear combination data structure, it would represent each
value of the dependent variable in terms of the difference between its actual
value and the value that would be estimated using the data structure.
These differences would then be encoded in an efficient information theoretic
way based on entropy.
}


% In this paper we develop an inexact minimum description length method for
% describing the structure of the data in terms of linear manifold clusters.
Our description language is the language of linear manifolds. Each linear
manifold cluster consists of the description of the linear manifold and
the coding of the data associated with the linear manifold is given by encoding
the orthogonal projection of each data point onto its manifold and the encoding
of the difference between its position off the manifold and its orthogonal
projection on the manifold. The inexactness of the description arises because
the description of the position of each data point off the manifold is
described not exactly, but with some controlled error.


Section \ref{sc:lit-review} is a literature review. Section \ref{sc:lmclus}
is a technical description of the linear manifold clustering stochastic search
technique. Section \ref{sc:mdl-lmclus} describes how the MDL principle is used
to determine whether to accept a cluster or not. Section \ref{sc:results}
discusses our results and section \ref{sc:conclusion} concludes the paper.
